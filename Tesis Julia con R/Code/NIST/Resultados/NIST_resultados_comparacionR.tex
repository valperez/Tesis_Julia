% Options for packages loaded elsewhere
\PassOptionsToPackage{unicode}{hyperref}
\PassOptionsToPackage{hyphens}{url}
%
\documentclass[
]{article}
\title{NIST\_comparacionR\_resultados}
\author{Valeria Perez}
\date{19/11/2021}

\usepackage{amsmath,amssymb}
\usepackage{lmodern}
\usepackage{iftex}
\ifPDFTeX
  \usepackage[T1]{fontenc}
  \usepackage[utf8]{inputenc}
  \usepackage{textcomp} % provide euro and other symbols
\else % if luatex or xetex
  \usepackage{unicode-math}
  \defaultfontfeatures{Scale=MatchLowercase}
  \defaultfontfeatures[\rmfamily]{Ligatures=TeX,Scale=1}
\fi
% Use upquote if available, for straight quotes in verbatim environments
\IfFileExists{upquote.sty}{\usepackage{upquote}}{}
\IfFileExists{microtype.sty}{% use microtype if available
  \usepackage[]{microtype}
  \UseMicrotypeSet[protrusion]{basicmath} % disable protrusion for tt fonts
}{}
\makeatletter
\@ifundefined{KOMAClassName}{% if non-KOMA class
  \IfFileExists{parskip.sty}{%
    \usepackage{parskip}
  }{% else
    \setlength{\parindent}{0pt}
    \setlength{\parskip}{6pt plus 2pt minus 1pt}}
}{% if KOMA class
  \KOMAoptions{parskip=half}}
\makeatother
\usepackage{xcolor}
\IfFileExists{xurl.sty}{\usepackage{xurl}}{} % add URL line breaks if available
\IfFileExists{bookmark.sty}{\usepackage{bookmark}}{\usepackage{hyperref}}
\hypersetup{
  pdftitle={NIST\_comparacionR\_resultados},
  pdfauthor={Valeria Perez},
  hidelinks,
  pdfcreator={LaTeX via pandoc}}
\urlstyle{same} % disable monospaced font for URLs
\usepackage[margin=1in]{geometry}
\usepackage{color}
\usepackage{fancyvrb}
\newcommand{\VerbBar}{|}
\newcommand{\VERB}{\Verb[commandchars=\\\{\}]}
\DefineVerbatimEnvironment{Highlighting}{Verbatim}{commandchars=\\\{\}}
% Add ',fontsize=\small' for more characters per line
\usepackage{framed}
\definecolor{shadecolor}{RGB}{248,248,248}
\newenvironment{Shaded}{\begin{snugshade}}{\end{snugshade}}
\newcommand{\AlertTok}[1]{\textcolor[rgb]{0.94,0.16,0.16}{#1}}
\newcommand{\AnnotationTok}[1]{\textcolor[rgb]{0.56,0.35,0.01}{\textbf{\textit{#1}}}}
\newcommand{\AttributeTok}[1]{\textcolor[rgb]{0.77,0.63,0.00}{#1}}
\newcommand{\BaseNTok}[1]{\textcolor[rgb]{0.00,0.00,0.81}{#1}}
\newcommand{\BuiltInTok}[1]{#1}
\newcommand{\CharTok}[1]{\textcolor[rgb]{0.31,0.60,0.02}{#1}}
\newcommand{\CommentTok}[1]{\textcolor[rgb]{0.56,0.35,0.01}{\textit{#1}}}
\newcommand{\CommentVarTok}[1]{\textcolor[rgb]{0.56,0.35,0.01}{\textbf{\textit{#1}}}}
\newcommand{\ConstantTok}[1]{\textcolor[rgb]{0.00,0.00,0.00}{#1}}
\newcommand{\ControlFlowTok}[1]{\textcolor[rgb]{0.13,0.29,0.53}{\textbf{#1}}}
\newcommand{\DataTypeTok}[1]{\textcolor[rgb]{0.13,0.29,0.53}{#1}}
\newcommand{\DecValTok}[1]{\textcolor[rgb]{0.00,0.00,0.81}{#1}}
\newcommand{\DocumentationTok}[1]{\textcolor[rgb]{0.56,0.35,0.01}{\textbf{\textit{#1}}}}
\newcommand{\ErrorTok}[1]{\textcolor[rgb]{0.64,0.00,0.00}{\textbf{#1}}}
\newcommand{\ExtensionTok}[1]{#1}
\newcommand{\FloatTok}[1]{\textcolor[rgb]{0.00,0.00,0.81}{#1}}
\newcommand{\FunctionTok}[1]{\textcolor[rgb]{0.00,0.00,0.00}{#1}}
\newcommand{\ImportTok}[1]{#1}
\newcommand{\InformationTok}[1]{\textcolor[rgb]{0.56,0.35,0.01}{\textbf{\textit{#1}}}}
\newcommand{\KeywordTok}[1]{\textcolor[rgb]{0.13,0.29,0.53}{\textbf{#1}}}
\newcommand{\NormalTok}[1]{#1}
\newcommand{\OperatorTok}[1]{\textcolor[rgb]{0.81,0.36,0.00}{\textbf{#1}}}
\newcommand{\OtherTok}[1]{\textcolor[rgb]{0.56,0.35,0.01}{#1}}
\newcommand{\PreprocessorTok}[1]{\textcolor[rgb]{0.56,0.35,0.01}{\textit{#1}}}
\newcommand{\RegionMarkerTok}[1]{#1}
\newcommand{\SpecialCharTok}[1]{\textcolor[rgb]{0.00,0.00,0.00}{#1}}
\newcommand{\SpecialStringTok}[1]{\textcolor[rgb]{0.31,0.60,0.02}{#1}}
\newcommand{\StringTok}[1]{\textcolor[rgb]{0.31,0.60,0.02}{#1}}
\newcommand{\VariableTok}[1]{\textcolor[rgb]{0.00,0.00,0.00}{#1}}
\newcommand{\VerbatimStringTok}[1]{\textcolor[rgb]{0.31,0.60,0.02}{#1}}
\newcommand{\WarningTok}[1]{\textcolor[rgb]{0.56,0.35,0.01}{\textbf{\textit{#1}}}}
\usepackage{longtable,booktabs,array}
\usepackage{calc} % for calculating minipage widths
% Correct order of tables after \paragraph or \subparagraph
\usepackage{etoolbox}
\makeatletter
\patchcmd\longtable{\par}{\if@noskipsec\mbox{}\fi\par}{}{}
\makeatother
% Allow footnotes in longtable head/foot
\IfFileExists{footnotehyper.sty}{\usepackage{footnotehyper}}{\usepackage{footnote}}
\makesavenoteenv{longtable}
\usepackage{graphicx}
\makeatletter
\def\maxwidth{\ifdim\Gin@nat@width>\linewidth\linewidth\else\Gin@nat@width\fi}
\def\maxheight{\ifdim\Gin@nat@height>\textheight\textheight\else\Gin@nat@height\fi}
\makeatother
% Scale images if necessary, so that they will not overflow the page
% margins by default, and it is still possible to overwrite the defaults
% using explicit options in \includegraphics[width, height, ...]{}
\setkeys{Gin}{width=\maxwidth,height=\maxheight,keepaspectratio}
% Set default figure placement to htbp
\makeatletter
\def\fps@figure{htbp}
\makeatother
\setlength{\emergencystretch}{3em} % prevent overfull lines
\providecommand{\tightlist}{%
  \setlength{\itemsep}{0pt}\setlength{\parskip}{0pt}}
\setcounter{secnumdepth}{-\maxdimen} % remove section numbering
\ifLuaTeX
  \usepackage{selnolig}  % disable illegal ligatures
\fi

\begin{document}
\maketitle

\hypertarget{problema}{%
\subsection{Problema}\label{problema}}

El problema a resolver es ajustar un polinomio de grado 10 a los datos
llamados filip proporcionados por NIST. De forma matricial, el problema
es \[y = X \beta\] donde la matrix \(X\) es de 82 x 11, la matriz
\(\beta\) es de 11 x 1. El problema es un problema de ecuaciones
lineales que intenté resolver con 4 métodos diferentes en Julia. Como no
encontraba el error a mis algoritmos, decidí ponerlos a prueba.
\textbackslash{} En general, son dos pruebas diferentes. La primera es
que comparo los resultados de los 4 algoritmos contra el resultado de la
función lm en R. La segunda es que hago las estimaciones de los
coeficientes \(\beta\) para todos los polinomios, desde el polinomio de
grado 1 hasta el polinomio de grado 10. \textbackslash{} Antes que nada,
el código para obtener los resultados. En primer lugar, leo y asigno a
las variables nuevas los resultados que obtuve de Julia.

\begin{Shaded}
\begin{Highlighting}[]
\FunctionTok{library}\NormalTok{(polynom)}
\end{Highlighting}
\end{Shaded}

\begin{verbatim}
## Warning: package 'polynom' was built under R version 4.0.5
\end{verbatim}

\begin{Shaded}
\begin{Highlighting}[]
\FunctionTok{library}\NormalTok{(knitr)}
\end{Highlighting}
\end{Shaded}

\begin{verbatim}
## Warning: package 'knitr' was built under R version 4.0.5
\end{verbatim}

\begin{Shaded}
\begin{Highlighting}[]
\CommentTok{\#setwd("\textasciitilde{}/ITAM/Tesis/Julia con R/Code/NIST")}
\NormalTok{data }\OtherTok{\textless{}{-}} \FunctionTok{read.csv}\NormalTok{(}\StringTok{"filip\_data.csv"}\NormalTok{)}

\CommentTok{\#setwd("\textasciitilde{}/ITAM/Tesis/Julia con R/Code/NIST/Resultados")}

\NormalTok{temp }\OtherTok{\textless{}{-}} \FunctionTok{list.files}\NormalTok{(}\AttributeTok{pattern =} \StringTok{"resultados\_grado\_"}\NormalTok{)}
\NormalTok{myfiles }\OtherTok{\textless{}{-}} \FunctionTok{lapply}\NormalTok{(temp, read.csv)}

\NormalTok{myfiles[[}\DecValTok{11}\NormalTok{]] }\OtherTok{\textless{}{-}}\NormalTok{ myfiles[[}\DecValTok{2}\NormalTok{]]}
\NormalTok{myfiles }\OtherTok{\textless{}{-}}\NormalTok{ myfiles[}\SpecialCharTok{{-}}\DecValTok{2}\NormalTok{]}

\ControlFlowTok{for}\NormalTok{ (i }\ControlFlowTok{in} \DecValTok{1}\SpecialCharTok{:}\DecValTok{10}\NormalTok{)\{}
  \FunctionTok{assign}\NormalTok{(}\FunctionTok{paste}\NormalTok{(}\StringTok{"resultados\_grado\_"}\NormalTok{, i), myfiles[[i]])}
\NormalTok{\}}
\end{Highlighting}
\end{Shaded}

Después, hago el código para el polinomio de grado 1.

\begin{Shaded}
\begin{Highlighting}[]
\CommentTok{\# Para polinomio de grado = 1}
\NormalTok{lm\_1 }\OtherTok{\textless{}{-}} \FunctionTok{lm}\NormalTok{(y }\SpecialCharTok{\textasciitilde{}}\NormalTok{ x, }\AttributeTok{data =}\NormalTok{ data, }\AttributeTok{x =} \ConstantTok{TRUE}\NormalTok{)}
\StringTok{\textasciigrave{}}\AttributeTok{resultados\_grado\_ 1}\StringTok{\textasciigrave{}}\SpecialCharTok{$}\NormalTok{R }\OtherTok{\textless{}{-}}\NormalTok{ lm\_1}\SpecialCharTok{$}\NormalTok{coefficients}
\FunctionTok{row.names}\NormalTok{(}\StringTok{\textasciigrave{}}\AttributeTok{resultados\_grado\_ 1}\StringTok{\textasciigrave{}}\NormalTok{) }\OtherTok{\textless{}{-}} \FunctionTok{c}\NormalTok{(}\StringTok{"b0"}\NormalTok{, }\StringTok{"b1"}\NormalTok{)}
\NormalTok{X\_1 }\OtherTok{\textless{}{-}}\NormalTok{ lm\_1}\SpecialCharTok{$}\NormalTok{x}
\end{Highlighting}
\end{Shaded}

Finalmente, el código para los polinomios de grado \textgreater{} 1

\begin{Shaded}
\begin{Highlighting}[]
\CommentTok{\# Para polinomios de grado \textgreater{} 1}

\ControlFlowTok{for}\NormalTok{ (i }\ControlFlowTok{in} \DecValTok{2}\SpecialCharTok{:}\DecValTok{10}\NormalTok{)\{}
  \CommentTok{\#Hacemos el modelo}
\NormalTok{  model }\OtherTok{\textless{}{-}} \FunctionTok{paste}\NormalTok{(}\StringTok{"y \textasciitilde{} x"}\NormalTok{, }\FunctionTok{paste}\NormalTok{(}\StringTok{"+ I(x\^{}"}\NormalTok{, }\DecValTok{2}\SpecialCharTok{:}\NormalTok{i, }\StringTok{")"}\NormalTok{, }\AttributeTok{sep=}\StringTok{\textquotesingle{}\textquotesingle{}}\NormalTok{, }\AttributeTok{collapse=}\StringTok{\textquotesingle{}\textquotesingle{}}\NormalTok{))}
  
  \CommentTok{\# Lo convertimos en formula}
\NormalTok{  form }\OtherTok{\textless{}{-}} \FunctionTok{formula}\NormalTok{(model)}
  
  \CommentTok{\#Ejecutamos el modelo}
\NormalTok{  lm.plus }\OtherTok{\textless{}{-}} \FunctionTok{lm}\NormalTok{(form, }\AttributeTok{data =}\NormalTok{ data, }\AttributeTok{x =} \ConstantTok{TRUE}\NormalTok{)}
  
  \CommentTok{\# Guardo el df correspondiente a un auxiliar}
\NormalTok{  resultados\_aux }\OtherTok{\textless{}{-}} \FunctionTok{get}\NormalTok{(}\FunctionTok{paste}\NormalTok{(}\StringTok{"resultados\_grado\_"}\NormalTok{, i))}
  \CommentTok{\# para unirle los coeficientes}
\NormalTok{  resultados\_aux}\SpecialCharTok{$}\NormalTok{R }\OtherTok{\textless{}{-}}\NormalTok{ lm.plus}\SpecialCharTok{$}\NormalTok{coefficients}
  
\NormalTok{  nombres }\OtherTok{\textless{}{-}} \FunctionTok{c}\NormalTok{(}\StringTok{"b0"}\NormalTok{)}
  \CommentTok{\# Para el nombre de los renglones}
  \ControlFlowTok{for}\NormalTok{ (k }\ControlFlowTok{in} \DecValTok{1}\SpecialCharTok{:}\NormalTok{i)\{}
\NormalTok{    nombres }\OtherTok{\textless{}{-}} \FunctionTok{c}\NormalTok{(nombres, }\FunctionTok{paste0}\NormalTok{(}\StringTok{"b"}\NormalTok{, k))}
\NormalTok{  \}}
  \FunctionTok{row.names}\NormalTok{(resultados\_aux) }\OtherTok{\textless{}{-}}\NormalTok{ nombres}
  
  \CommentTok{\#Finalmente, hago el df final}
  \FunctionTok{assign}\NormalTok{(}\FunctionTok{paste}\NormalTok{(}\StringTok{"resultados\_grado\_"}\NormalTok{, i), resultados\_aux)}
  
  \FunctionTok{assign}\NormalTok{(}\FunctionTok{paste}\NormalTok{(}\StringTok{"X\_"}\NormalTok{, i), lm.plus}\SpecialCharTok{$}\NormalTok{x)}
  
\NormalTok{\}}
\end{Highlighting}
\end{Shaded}

Los resultados estan en las siguientes tablas.

\begin{Shaded}
\begin{Highlighting}[]
\FunctionTok{kable}\NormalTok{(}\StringTok{\textasciigrave{}}\AttributeTok{resultados\_grado\_ 1}\StringTok{\textasciigrave{}}\NormalTok{, }\AttributeTok{caption =} \StringTok{"Polinomio grado 1"}\NormalTok{)}
\end{Highlighting}
\end{Shaded}

\begin{longtable}[]{@{}lrrrrr@{}}
\caption{Polinomio grado 1}\tabularnewline
\toprule
& PolFit & QRPivot & MoorePenrose & LinearFit & R \\
\midrule
\endfirsthead
\toprule
& PolFit & QRPivot & MoorePenrose & LinearFit & R \\
\midrule
\endhead
b0 & 1.0592655 & 1.0592655 & 1.0592655 & 1.0592655 & 1.0592655 \\
b1 & 0.0340946 & 0.0340946 & 0.0340946 & 0.0340946 & 0.0340946 \\
\bottomrule
\end{longtable}

\begin{Shaded}
\begin{Highlighting}[]
\FunctionTok{kable}\NormalTok{(}\StringTok{\textasciigrave{}}\AttributeTok{resultados\_grado\_ 2}\StringTok{\textasciigrave{}}\NormalTok{, }\AttributeTok{caption =} \StringTok{"Polinomio grado 2"}\NormalTok{)}
\end{Highlighting}
\end{Shaded}

\begin{longtable}[]{@{}lrrrrr@{}}
\caption{Polinomio grado 2}\tabularnewline
\toprule
& PolFit & QRPivot & MoorePenrose & LinearFit & R \\
\midrule
\endfirsthead
\toprule
& PolFit & QRPivot & MoorePenrose & LinearFit & R \\
\midrule
\endhead
b0 & 0.9223409 & 0.9223409 & 0.9223409 & 0.9223409 & 0.9223409 \\
b1 & -0.0140724 & -0.0140724 & -0.0140724 & -0.0140724 & -0.0140724 \\
b2 & -0.0039770 & -0.0039770 & -0.0039770 & -0.0039770 & -0.0039770 \\
\bottomrule
\end{longtable}

\begin{Shaded}
\begin{Highlighting}[]
\FunctionTok{kable}\NormalTok{(}\StringTok{\textasciigrave{}}\AttributeTok{resultados\_grado\_ 3}\StringTok{\textasciigrave{}}\NormalTok{, }\AttributeTok{caption =} \StringTok{"Polinomio grado 3"}\NormalTok{)}
\end{Highlighting}
\end{Shaded}

\begin{longtable}[]{@{}lrrrrr@{}}
\caption{Polinomio grado 3}\tabularnewline
\toprule
& PolFit & QRPivot & MoorePenrose & LinearFit & R \\
\midrule
\endfirsthead
\toprule
& PolFit & QRPivot & MoorePenrose & LinearFit & R \\
\midrule
\endhead
b0 & 0.3902712 & 0.3902712 & 0.3902712 & 0.3902712 & 0.3902712 \\
b1 & -0.3033645 & -0.3033645 & -0.3033645 & -0.3033645 & -0.3033645 \\
b2 & -0.0537195 & -0.0537195 & -0.0537195 & -0.0537195 & -0.0537195 \\
b3 & -0.0027265 & -0.0027265 & -0.0027265 & -0.0027265 & -0.0027265 \\
\bottomrule
\end{longtable}

\begin{Shaded}
\begin{Highlighting}[]
\FunctionTok{kable}\NormalTok{(}\StringTok{\textasciigrave{}}\AttributeTok{resultados\_grado\_ 4}\StringTok{\textasciigrave{}}\NormalTok{, }\AttributeTok{caption =} \StringTok{"Polinomio grado 4"}\NormalTok{)}
\end{Highlighting}
\end{Shaded}

\begin{longtable}[]{@{}lrrrrr@{}}
\caption{Polinomio grado 4}\tabularnewline
\toprule
& PolFit & QRPivot & MoorePenrose & LinearFit & R \\
\midrule
\endfirsthead
\toprule
& PolFit & QRPivot & MoorePenrose & LinearFit & R \\
\midrule
\endhead
b0 & 2.6444057 & 2.6444057 & 2.6444057 & 2.6444057 & 2.6444057 \\
b1 & 1.3744058 & 1.3744058 & 1.3744058 & 1.3744058 & 1.3744058 \\
b2 & 0.3970969 & 0.3970969 & 0.3970969 & 0.3970969 & 0.3970969 \\
b3 & 0.0492439 & 0.0492439 & 0.0492439 & 0.0492439 & 0.0492439 \\
b4 & 0.0021749 & 0.0021749 & 0.0021749 & 0.0021749 & 0.0021749 \\
\bottomrule
\end{longtable}

\begin{Shaded}
\begin{Highlighting}[]
\FunctionTok{kable}\NormalTok{(}\StringTok{\textasciigrave{}}\AttributeTok{resultados\_grado\_ 5}\StringTok{\textasciigrave{}}\NormalTok{, }\AttributeTok{caption =} \StringTok{"Polinomio grado 5"}\NormalTok{)}
\end{Highlighting}
\end{Shaded}

\begin{longtable}[]{@{}lrrrrr@{}}
\caption{Polinomio grado 5}\tabularnewline
\toprule
& PolFit & QRPivot & MoorePenrose & LinearFit & R \\
\midrule
\endfirsthead
\toprule
& PolFit & QRPivot & MoorePenrose & LinearFit & R \\
\midrule
\endhead
b0 & 4.3006539 & 4.3006539 & 4.3006539 & 4.3006544 & 4.3006539 \\
b1 & 2.9237727 & 2.9237727 & 2.9237727 & 2.9237731 & 2.9237727 \\
b2 & 0.9589165 & 0.9589165 & 0.9589165 & 0.9589167 & 0.9589165 \\
b3 & 0.1481183 & 0.1481183 & 0.1481183 & 0.1481184 & 0.1481183 \\
b4 & 0.0106384 & 0.0106384 & 0.0106384 & 0.0106384 & 0.0106384 \\
b5 & 0.0002825 & 0.0002825 & 0.0002825 & 0.0002825 & 0.0002825 \\
\bottomrule
\end{longtable}

\begin{Shaded}
\begin{Highlighting}[]
\FunctionTok{kable}\NormalTok{(}\StringTok{\textasciigrave{}}\AttributeTok{resultados\_grado\_ 6}\StringTok{\textasciigrave{}}\NormalTok{, }\AttributeTok{caption =} \StringTok{"Polinomio grado 6"}\NormalTok{)}
\end{Highlighting}
\end{Shaded}

\begin{longtable}[]{@{}lrrrrr@{}}
\caption{Polinomio grado 6}\tabularnewline
\toprule
& PolFit & QRPivot & MoorePenrose & LinearFit & R \\
\midrule
\endfirsthead
\toprule
& PolFit & QRPivot & MoorePenrose & LinearFit & R \\
\midrule
\endhead
b0 & -18.0975496 & -18.0975496 & -18.0975496 & 1.9043149 &
-18.0975496 \\
b1 & -22.2966441 & -22.2966441 & -22.2966441 & 0.0000000 &
-22.2966441 \\
b2 & -10.5769427 & -10.5769427 & -10.5769427 & -0.4810935 &
-10.5769427 \\
b3 & -2.5981095 & -2.5981095 & -2.5981095 & -0.2185587 & -2.5981095 \\
b4 & -0.3486584 & -0.3486584 & -0.3486584 & -0.0403353 & -0.3486584 \\
b5 & -0.0242444 & -0.0242444 & -0.0242444 & -0.0033915 & -0.0242444 \\
b6 & -0.0006834 & -0.0006834 & -0.0006834 & -0.0001075 & -0.0006834 \\
\bottomrule
\end{longtable}

\begin{Shaded}
\begin{Highlighting}[]
\FunctionTok{kable}\NormalTok{(}\StringTok{\textasciigrave{}}\AttributeTok{resultados\_grado\_ 7}\StringTok{\textasciigrave{}}\NormalTok{, }\AttributeTok{caption =} \StringTok{"Polinomio grado 7"}\NormalTok{)}
\end{Highlighting}
\end{Shaded}

\begin{longtable}[]{@{}lrrrrr@{}}
\caption{Polinomio grado 7}\tabularnewline
\toprule
& PolFit & QRPivot & MoorePenrose & LinearFit & R \\
\midrule
\endfirsthead
\toprule
& PolFit & QRPivot & MoorePenrose & LinearFit & R \\
\midrule
\endhead
b0 & -8.6609575 & -8.6609575 & -8.6609588 & 0.0000000 & -8.6609575 \\
b1 & -9.8263025 & -9.8263025 & -9.8263040 & 0.0000000 & -9.8263025 \\
b2 & -3.6650346 & -3.6650346 & -3.6650351 & 0.0000000 & -3.6650346 \\
b3 & -0.5141292 & -0.5141292 & -0.5141295 & -0.1742561 & -0.5141292 \\
b4 & 0.0207340 & 0.0207340 & 0.0207339 & -0.0871662 & 0.0207340 \\
b5 & 0.0142807 & 0.0142807 & 0.0142807 & -0.0170740 & 0.0142807 \\
b6 & 0.0015076 & 0.0015076 & 0.0015076 & -0.0015216 & 0.0015076 \\
b7 & 0.0000525 & 0.0000525 & 0.0000525 & -0.0000515 & 0.0000525 \\
\bottomrule
\end{longtable}

\begin{Shaded}
\begin{Highlighting}[]
\FunctionTok{kable}\NormalTok{(}\StringTok{\textasciigrave{}}\AttributeTok{resultados\_grado\_ 8}\StringTok{\textasciigrave{}}\NormalTok{, }\AttributeTok{caption =} \StringTok{"Polinomio grado 8"}\NormalTok{)}
\end{Highlighting}
\end{Shaded}

\begin{longtable}[]{@{}lrrrrr@{}}
\caption{Polinomio grado 8}\tabularnewline
\toprule
& PolFit & QRPivot & MoorePenrose & LinearFit & R \\
\midrule
\endfirsthead
\toprule
& PolFit & QRPivot & MoorePenrose & LinearFit & R \\
\midrule
\endhead
b0 & 175.9750151 & 175.9750152 & 175.9750570 & 0.0000000 &
175.9750151 \\
b1 & 269.2657673 & 269.2657674 & 269.2657897 & 0.0000000 &
269.2657673 \\
b2 & 177.4702511 & 177.4702512 & 177.4702530 & 0.0000000 &
177.4702511 \\
b3 & 65.4365443 & 65.4365443 & 65.4365573 & 0.0000000 & 65.4365443 \\
b4 & 14.7617342 & 14.7617342 & 14.7617357 & 0.0554432 & 14.7617342 \\
b5 & 2.0867401 & 2.0867401 & 2.0867399 & 0.0285355 & 2.0867401 \\
b6 & 0.1806048 & 0.1806048 & 0.1806048 & 0.0056228 & 0.1806048 \\
b7 & 0.0087567 & 0.0087567 & 0.0087567 & 0.0004978 & 0.0087567 \\
b8 & 0.0001823 & 0.0001823 & 0.0001823 & 0.0000166 & 0.0001823 \\
\bottomrule
\end{longtable}

\begin{Shaded}
\begin{Highlighting}[]
\FunctionTok{kable}\NormalTok{(}\StringTok{\textasciigrave{}}\AttributeTok{resultados\_grado\_ 9}\StringTok{\textasciigrave{}}\NormalTok{, }\AttributeTok{caption =} \StringTok{"Polinomio grado 9"}\NormalTok{)}
\end{Highlighting}
\end{Shaded}

\begin{longtable}[]{@{}lrrrrr@{}}
\caption{Polinomio grado 9}\tabularnewline
\toprule
& PolFit & QRPivot & MoorePenrose & LinearFit & R \\
\midrule
\endfirsthead
\toprule
& PolFit & QRPivot & MoorePenrose & LinearFit & R \\
\midrule
\endhead
b0 & -174.2804420 & -174.2804443 & -174.2802936 & 0.0000000 &
-174.2804456 \\
b1 & -326.8822038 & -326.8822077 & -326.8825148 & 0.0000000 &
-326.8822099 \\
b2 & -266.0565360 & -266.0565388 & -266.0571501 & 0.0000000 &
-266.0565405 \\
b3 & -123.9216126 & -123.9216137 & -123.9219121 & 0.0000000 &
-123.9216144 \\
b4 & -36.3816705 & -36.3816708 & -36.3816855 & 0.0000000 &
-36.3816710 \\
b5 & -6.9791883 & -6.9791883 & -6.9791852 & -0.0151684 & -6.9791884 \\
b6 & -0.8746602 & -0.8746602 & -0.8746601 & -0.0078823 & -0.8746602 \\
b7 & -0.0690601 & -0.0690601 & -0.0690601 & -0.0015521 & -0.0690601 \\
b8 & -0.0031183 & -0.0031183 & -0.0031183 & -0.0001365 & -0.0031183 \\
b9 & -0.0000614 & -0.0000614 & -0.0000614 & -0.0000045 & -0.0000614 \\
\bottomrule
\end{longtable}

\begin{Shaded}
\begin{Highlighting}[]
\FunctionTok{kable}\NormalTok{(}\StringTok{\textasciigrave{}}\AttributeTok{resultados\_grado\_ 10}\StringTok{\textasciigrave{}}\NormalTok{, }\AttributeTok{caption =} \StringTok{"Polinomio grado 10"}\NormalTok{)}
\end{Highlighting}
\end{Shaded}

\begin{longtable}[]{@{}lrrrrr@{}}
\caption{Polinomio grado 10}\tabularnewline
\toprule
& PolFit & QRPivot & MoorePenrose & LinearFit & R \\
\midrule
\endfirsthead
\toprule
& PolFit & QRPivot & MoorePenrose & LinearFit & R \\
\midrule
\endhead
b0 & -1467.4896771 & 9.0134262 & 9.1661966 & 0.0000000 & -174.2804456 \\
b1 & -2772.1797121 & 1.6525458 & 2.7223863 & 0.0000000 & -326.8822099 \\
b2 & -2316.3711835 & -5.7676064 & -4.2207903 & 0.0000000 &
-266.0565405 \\
b3 & -1127.9739915 & -3.8636656 & -2.7920321 & 0.0000000 &
-123.9216144 \\
b4 & -354.4782499 & -0.6703657 & -0.2334151 & 0.0000000 & -36.3816710 \\
b5 & -75.1242053 & 0.1806044 & 0.2944490 & 0.0000000 & -6.9791884 \\
b6 & -10.8753186 & 0.1055234 & 0.1251241 & 0.0036864 & -0.8746602 \\
b7 & -1.0622150 & 0.0214449 & 0.0236685 & 0.0019173 & -0.0690601 \\
b8 & -0.0670191 & 0.0022775 & 0.0024377 & 0.0003759 & -0.0031183 \\
b9 & -0.0024678 & 0.0001262 & 0.0001329 & 0.0000328 & -0.0000614 \\
b10 & -0.0000403 & 0.0000029 & 0.0000030 & 0.0000011 & NA \\
\bottomrule
\end{longtable}

\end{document}
