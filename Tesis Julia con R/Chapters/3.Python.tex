\chapter{Python}

\say{Python es un lenguaje de programación que te permite trabajar rápido e integrar sistemas más eficientemente} es la primera frase que se lee en la página oficial de \textsf{Python} \url{https://www.python.org/}. Guido van Rossum comenzó a crear el lenguaje a finales de los ochentas, pero lo hizo público hasta 1991. Empresas importantes como Youtube y Google han elogiado \textsf{Python} por su rapidez y constante desarrollo. Es un lenguaje más antiguo que \textsf{Julia} y con mayor popularidad. Consecuentemente, hay muchos videos, artículos, blogs y libros sobre su uso y desarrollo. Se decidió incluirlo en esta tesis ya que se considera es un excelente punto de comparación con \textsf{Julia} no solo en rapidez sino también en la sencillez y facilidad de programación. En este capítulo se explican los paquetes principales y la interfaz que se utilizó. 

\section{Listas}
\say{Una \textit{lista} es una colección de elementos en un orden particular}, \cite{matthes2019python}. Las listas son el objeto principal y más elemental de \textsf{Python}. Son una estructura de datos por lo que se usan para almacenar varios elementos en una sola variable. Las listas se crean usando paréntesis cuadrados \texttt{[]}. Por ejemplo, si se quisiera hacer una lista de animales en el zoológico el comando sería 

\valinline{Cambie el paquete que use para que se cambiara el font, que tal?}

\begin{lstlisting}[language=Python]
	animales = ["zebra", "leon", "jirafa", "elefante"]
	animales[1]
\end{lstlisting}

Para accesar a un elemento de la lista hay que usar los paréntesis cuadrados. Por ejemplo, \texttt{animales[1]} me regresa \texttt{"león"}. Una característica clave de las listas en \textsf{Python} es que, a diferencia de \textsf{R} y \textsf{Julia}, las listas comienzan a numerar sus elementos desde el cero. 

En esta tesis se utilizaron las listas como estructura de datos ya que están ordenadas, pueden ser cambiadas y permiten valores duplicados. Por lo tanto, son fáciles y eficientes para trabajar. Como cada estructura, las listas tienen sus propios métodos que vienen en listas y explicados en la documentación de \textsf{Python}, \cite{doc_python}. 


\section{Paquetes}
Al igual que \textsf{Julia}, \textsf{Python} tiene diversidad de paquetes para hacer todo tipo de análisis. Para usar cualquier instrucción de un paquete se tiene que primero nombrar su apodo y después llamar a la función. El apodo del paquete se lo otorga el usuario al momento de importarlo. Por ejemplo, 
\begin{lstlisting}[language=Python]
	import numpy as np
\end{lstlisting}

\noindent importa el paquete \texttt{NumPy} con el apodo \texttt{np}. Si se quisiera llamar a la función \texttt{array} de este paquete se tendría que escribir el comando \texttt{np.array}. Esto podría parecer tedioso, pero lo considero una ventaja ya que siempre sabes el paquete que estás usando. 

\subsection{NumPy} \label{seccion_numpy}
NumPy es el paquete fundamental para computación científica en \textsf{Python} ya que proporciona los objetos de matriz multidimensional.
Hay varias diferentes entre matrices NumPy y secuencias del \textsf{Python} estándar. Algunas de ellas son que los arreglos de NumPy tienen dimensiones fijas en su creación que no se pueden cambiar; sus elementos deben ser del mismo tipo de dato; facilitan operaciones matemáticas en grandes cantidades de datos; y, finalmente, una gran parte de la comunidad que utiliza \textsf{Python} también usa arreglos de NumPy \cite{numpy_manual}. \val{Esto es parafraseado, está bien referenciado?}

En esta tesis se usó NumPy para crear y manipular arreglos así como hacer un ajuste polinomial de mínimos cuadrados. A continuación está la lista completa de comandos que se utilizaron con su explicación que se obtuvo del paquete oficial de \texttt{NumPy}, \cite{numpy_manual}. 

\begin{itemize}
	\item np.array([\texttt{lista}]): Crea un arreglo con los valores de la \texttt{lista}.
	
	\item np.insert(\texttt{arr, obj, values}): Inserta los valores \texttt{values} en el arreglo \texttt{arr} antes de los índices \texttt{obj}.
	
	\item np.arange(\texttt{start, stop}): Crea un arreglo con valores espaciados uniformemente desde \texttt{start} hasta el número antes de \texttt{stop}. 
	
	\item np.transpose(\texttt{a}): Transpone el objeto a\texttt{a}.
	
	\item np.concatenate($a_1, a_2, \dots $): Une la secuencia de arreglos en uno ya existente.
	
	\item np.ones(\texttt{shape}): Crea una matriz de tamaño \texttt{shape} llena con unos.
	
	\item np.diag(\texttt{v}): Extrae la diagonal de la matriz \texttt{v} o crea una matriz diagonal de tamaño \texttt{v}.
	
	\item np.linalg.inv(\texttt{a}): Calcula la inversa multiplicativa de la matriz \texttt{v}. 
	
	\item np.random.choice(\texttt{a, size = None, replace = True, p = None}): Genera una muestra aleatoria de \texttt{a} de tamaño \texttt{size} con o sin reemplazo. 
	
	\item np.polyfit(\texttt{x, y, deg}): Hace un ajuste polinomial de grado \texttt{deg} a los puntos \texttt{(x, y)} usando el método de mínimos cuadrados. 
	
\end{itemize}


\subsection{pandas}
pandas es el segundo paquete primordial y básico de \textsf{Python} ya que se enfoca en la manipulación y análisis de datos. Sus funciones se enfocan en el uso eficiente de dataframes, leer y escribir datos, agrupación y unión de varios conjuntos de datos, entre otros \cite{pandas_manual}. En esta tesis utilice los siguientes comandos de pandas. 

\begin{itemize}
	\item pd.read\_csv(\texttt{filepath}): Lee un archivo \texttt{csv} y lo convierte a DataFrame.
	
	\item pd.DataFrame(\texttt{data}): Crea un objeto de tipo dataframe con los datos \texttt{data}. 
	
	\item pd.get\_dummies(\texttt{data}): Convierte variables categóricas \texttt{data} en variables indicadoras o dummie. 
\end{itemize}

\subsection{os}
Otro paquete que se utilizó en este trabajo fue \texttt{os} ya que proporciona una manera de usar funcionalidad dependiente del sistema operativo \cite{doc_python} \val{ portable way of using operating system dependent functionality}. En otras palabras es el paquete que permite hacer la conexión entre \textsf{Python} y los archivos de una computadora. Los comandos de este paquete que se utilizaron son dos. El primero fue \texttt{os.chdir(path)} que permite seleccionar el directorio en el que se está trabajando. El segundo fue \texttt{os.listdir(path)} que proporciona una lista de archivos en el \texttt{path} dado. 


\subsection{scikit-learn}
scikit-learn es un paquete creado para hacer \textit{machine learning} o aprendizaje de máquina en \textsf{Python}. También es conocido como \texttt{sklearn} y proporciona herramientas simples y eficientes para la predicción en análisis de datos. Sus herramientas hacen clasificación, regresión, \textit{clustering} o agrupamiento, reducción de dimensiones y selección de modelos \cite{doc_python}.

Para este trabajo se utilizó la parte de regresiones lineales del paquete. El usuario puede importar el paquete  de dos maneras. 

\begin{lstlisting}[language=Python]
	import sklearn
	from sklearn import linear_model
	
	regr = linear_model.LinearRegression()
	model = regr.fit(x, y)
\end{lstlisting}

El primer comando importa el paquete completo, mientras que el segundo solo importa la parte de modelos lineales. Con el paquete cargado, la tercera línea de código se encarga de guardar en la variable \texttt{reg} que se busca ajustar un modelo linear definido como la ecuación \ref{eq_rlm} usando el método de mínimos cuadrados. Después, el último comando del código calcula los coeficientes $\beta$. 

\subsection{itertools}
\say{Itertools es un módulo que estandariza un conjunto núcleo de herramientas rápidas y eficientes de memoria que son útiles en sí mismas o en combinación}, \cite{doc_python}. Algunas de las herramientas que tiene este módulo se pueden recrear sin la necesidad del mismo, pero la ventaja de utilizar \texttt{itertools} es la velocidad en la que las genera. 

En este trabajo se utilizo \texttt{itertools.combinations()} para crear las combinaciones de posibles factores activos del problema del capítulo \ref{chapter_MDopt}. Todos los paquetes anteriores se implementaron en la interfaz gráfica \textsf{Jupyter} que se introduce a continuación.

\section{Jupyter} \label{cap_jupyter}
\say{Jupyter Notebook es la aplicación web original para crear y compartir documentos computacionales. Es un programa que existe para desarrollar software de manera pública en decenas de lenguajes de programación incluyendo \textsf{R}, \textsf{Python} y \textsf{Julia}}, \cite{jupyter_page}. 

La manera sencilla de obtener \textsf{Jupyter} es instalando \textsf{Anaconda}. \textsf{Anaconda} es una interfaz gráfica que permite manejar y administrar aplicaciones, paquetes, ambientes y canales sin necesidad de usar comandos en el \texttt{cmd}. Para instalar Anaconda en Windows se debe ir a la página \url{https://docs.anaconda.com/anaconda/install/windows/} y seguir las instrucciones de instalación. Esto puede tomar unos minutos. 

La versatilidad de \textsf{Jupyter} en los tres lenguajes es la razón principal por la que se decidió usarlo en esta tesis. Poder usar los tres lenguajes en un mismo software permitió tener una mejor organización y una traducción entre lenguajes fluida. 

Uno de los prerequisitos para instalar Jupyter es tener \textsf{Python}. Por lo tanto, este lenguaje que ya viene sin necesidad de ninguna otra instalación. El caso de \textsf{R} y \textsf{Julia} no es igual. En las siguientes secciones se explica su instalación.

\subsection{Julia}

El primer paso es haber instalado \textsf{Julia}. Después, se debe instalar el paquete \texttt{IJulia} usando los pasos descritos en \ref{instalacion_paquete}. Esto solo se tiene que hacer una vez. Para confirmar que la instalación esté bien hecha se debe abrir \textsf{Jupyter}, seleccionar \textsf{New} y debe aparecer la opción de \textsf{Julia 1.6.3} (o la versión de \textsf{Julia} que esté instalada en la computadora). 

\subsection{R}

Hay varias maneras de instalar \textsf{R} en \textsf{Jupyter}, pero se expondrá la forma descrita en el manual de \textsf{Anaconda}, \cite{anaconda_doc}. 

\begin{enumerate}
	\item Abrir el Navegador de \textsf{Anaconda} (no confundir con el de \textsf{Jupyter Notebook}).
	
	\item Seleccionar \texttt{Environments} y después la opción de \texttt{Create} ubicada en la esquina inferior izquierda. 
	
	\item Aparecerá una ventana donde permite nombrar el \texttt{Environment} como se prefiera. Se debe seleccionar la versión de \textsf{Python} que se tenga y seleccionar la casilla al lado de \textsf{R}. Después, se debe pulsar la opción de \texttt{Create}. 
	
	\item Para usar el ambiente que se acaba de crear en \textsf{Jupyter} se selecciona la flecha de lado derecho del nombre del ambiente nuevo. Entre las opciones seleccionar la opción de \texttt{Open with Jupyter Notebook}. 
	
	\item Por último, se debe seleccionar el botón de \texttt{New} y después \textsf{R} para crear un archivo que trabaje con \textsf{R}. 
\end{enumerate}
