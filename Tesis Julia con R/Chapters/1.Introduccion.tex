\chapter{Introducción}

\textit{\say{I always promote \textsf{Julia} among friends and colleagues in Latin America, even when it has been difficult to convince them because of the scarce resources of \textsf{Julia} in Spanish. I firmly believe in open access knowledge without barriers (either language barriers, accessibility, or others), and I will always advocate for that}},\footnote{Yo siempre promuevo Julia entre amigos y colegas de América Latina aún cuando ha sido difícil convencerlos por la escasez de recursos de Julia en español. Creo firmemente en el libre acceso al conocimiento sin barreras (ya sean lingüísticas, de accesibilidad u otras) y siempre defenderé eso.} \cite{articulo_10anos}. Las palabras de la chilena Pamela Bustamente, usuaria de \textsf{Julia},  engloban el motivo del desarrollo de esta tesis.  

Mi camino con \textsf{Julia} comenzó a principios del 2021 cuando tuve la oportunidad de trabajar en el Instituto Mexicano del Seguro Social (IMSS). \textsf{Julia} fue la herramienta que utilice para desarrollar un proyecto fundamentado en estadística bayesiana y requería de una gran cantidad de simulaciones. Tardé poco tiempo en encontrarme con las dificultades que menciona Pamela y algunas más. Sin embargo, \textsf{Julia} debe tener otras cualidades que frecuentemente lo destaquen como un lenguaje prometedor que cada día va tomando más fuerza en la comunidad de programadores. 

Al principio, dichas cualidades eran un misterio para mí. Mi interrogante principal fue sobre la necesidad de crear un nuevo lenguaje. ¿Por qué usar \textsf{Julia} y no \textsf{Python} o \textsf{R}?, ¿Cuál fue la motivación de su creación? y, después de encontrarme con una falta de recursos, ¿Cómo es posible que 10 años más tarde exista tan poca ayuda de este lenguaje?  Esta tesis es mi esfuerzo por mostrar un nuevo lenguaje, sus alcances y hacer una comparativa con el conocimiento ya existente. Asimismo, mi trabajo queda como invitación a futuros usuarios y usuarias hispanohablantes de utilizar \textsf{Julia}.

Este trabajo no es un manual de \textsf{Julia} ni de ningún otro lenguaje. Eso ya existe. Lo que se busca es explicar pros y contras que se encontraron al utilizar \textsf{Julia}, \textsf{Python} y \textsf{R} en tres ejercicios distintos. En cada uno de los ejemplos se busca mostrar dos aspectos principales en los lenguajes: la capacidad y la velocidad de cómputo. El primer término se refiere al nivel de exactitud que presentan los lenguajes al calcular la solución de los ejercicios. Por otro lado, cuando se habla de velocidad de cómputo, se hace referencia al tiempo que el usuario o la usuaria deben esperar para que el lenguaje ejecute el código y muestre la solución calculada. El título de la tesis se refiere justamente a dicha comparación, ya que se omite mencionar teoría de cómputo estadístico en sí. Asimismo, \textsf{Python} y \textsf{R} se eligieron por su uso y extenso desarrollo en la ciencia de datos, el tema principal de este trabajo. Asimismo, el uso de los tres lenguajes permite exponer la experiencia de usuario al trabajar con los tres lenguajes. 

La tesis se divide en dos partes. El propósito de la primera parte es proveer una imagen general de los tres lenguajes, así como las funciones que se utilizaron en la segunda parte. Primero, se expone la instalación de \textsf{Julia} en un sistema operativo \textsf{Windows} para después explicar aspectos básicos del lenguaje. Asimismo, se da una introducción a dos paquetes fundamentales para este trabajo. Esto se hace pensando que \textsf{Julia} es el lenguaje más reciente y se busca que el lector navegue fácilmente por el código mostrado. Después, se presentan los paquetes y funciones que se utilizaron en \textsf{Python} y en \textsf{R} de manera más general suponiendo que el lector ya está familiarizado con ellos. 

La segunda parte de la tesis consta de tres proyectos cuyo objetivo es mostrar un aspecto diferente en los lenguajes. El primer proyecto toma datos del National Institute of Standards and Technology (NIST) para medir la precisión numérica de cada lenguaje al hacer el ajuste de un polinomio de grado 10. El segundo ejercicio usa los datos del Censo de Población y Vivienda de México del 2020 realizado por el Instituto Nacional de Estadística y Geografía (INEGI). El objetivo de este ejercicio es el manejo y manipulación de una gran cantidad de datos. Finalmente, el tercer proyecto presenta la programación de un algoritmo de búsqueda que se utiliza en la discriminación de modelos en diseños de experimentos. El enfoque de este ejercicio es la búsqueda de modelos, no el diseño de experimentos. Los cálculos son más intensivos por lo que el ejemplo busca medir la capacidad y velocidad de cómputo de los lenguajes. 

En suma, el capítulo \ref{cap_julia} presenta una primera introducción a \textsf{Julia}. Los capítulos \ref{cap_python} y \ref{cap_R} describen los paquetes y funciones utilizados en \textsf{Python} y \textsf{R}, respectivamente. El capítulo \ref{cap_polinomios} muestra la primera aplicación de los lenguajes al resolver un problema propuesto por NIST. Posteriormente, en el capítulo \ref{cap_rlm} se presenta el uso de modelos de regresión lineal en una gran cantidad de datos. El capítulo \ref{chapter_MDopt} expone un algoritmo utilizado para la búsqueda de modelos en el contexto de diseño de experimentos. Finalmente, en el capítulo \ref{cap_conclusiones} se presenta un resumen del trabajo así como la experiencia de usuario y opinión personal acerca del uso de \textsf{Julia, Python} y \textsf{R}. 

A continuación, se comienza este trabajo con la presentación de \textsf{Julia}. 



