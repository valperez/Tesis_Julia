\chapter{Introducción}

\say{I always promote \textsf{Julia} among friends and colleagues in Latin America, even when it has been difficult to convince them because of the scarce resources of \textsf{Julia} in Spanish. I firmly believe in open access knowledge without barriers (either language barriers, accessibility, or others), and I will always advocate for that} \cite{articulo_10anos}. Las palabras de la chilena Pamela Bustamente, usuaria de \textsf{Julia},  engloban la razón de ser de esta tesis. 

Mi camino con \textsf{Julia} comenzó a principios del 2021 cuando tuve la oportunidad de trabajar en el Instituto Mexicano del Seguro Social (IMSS). \textsf{Julia} fue la herramienta que utilice para desarrollar un proyecto que estaba fundamentado en estadística bayesiana y requería de una gran cantidad de simulaciones. No tarde mucho tiempo en encontrarme con las dificultades que menciona Pamela y algunas más. Sin embargo, \textsf{Julia} debe tener otras cualidades que frecuentemente lo destaquen como un lenguaje prometedor que cada día va tomando más fuerza en la comunidad de programadores. 

Al principio, dichas cualidades eran un misterio para mí. Mi interrogante principal fue sobre la necesidad de crear este nuevo lenguaje. ¿Por qué usar \textsf{Julia} y no \textsf{Python} o \textsf{R}?, ¿Cuál fue la motivación de su creación? y, después de encontrarme con una falta de recursos, ¿Cómo es posible que 10 años más tarde hay tan poca ayuda de este lenguaje?  Esta tesis es mi esfuerzo por mostrar un nuevo lenguaje, sus alcances y hacer una comparativa con lo que ya se conoce. De paso, mi trabajo queda como evidencia y punto de partida para futuros usuarios hispanohablantes.  

Este trabajo no es un manual de \textsf{Julia} ni de ningún otro lenguaje. Eso ya existe. Lo que se busca es explicar pros y contras que se encontraron al utilizar \textsf{Julia}, \textsf{Python} y \textsf{R} en tres ejercicios distintos.

La tesis se divide en dos partes. El propósito de la primera parte es dar una imagen general de las funciones que se utilizaron en los tres lenguajes para crear la segunda parte. Primero, se expone la instalación de \textsf{Julia} en un sistema operativo \textsf{Windows} para después explicar aspectos básicos del lenguaje. También, se da una introducción a dos paquetes fundamentales para este trabajo. Esto se hace pensando que \textsf{Julia} es el lenguaje más reciente y se busca que el lector navegue fácilmente por el código presentado. Después, se presentan los paquetes y funciones que se utilizaron en \textsf{Python} y en \textsf{R} suponiendo que el lector ya está familiarizado con ellos. 

La segunda parte de la tesis consta de tres ejercicios cuyo objetivo es mostrar un aspecto diferente en los lenguajes. El primer ejercicio toma datos del National Institute of Standards and Technology (NIST) para medir la precisión numérica de cada lenguaje al hacer el ajuste de un polinomio de grado 10. El segundo ejercicio usa los datos del Censo de Población y Vivienda de México del 2020 hecho por el Instituto Nacional de Estadística y Geografía (INEGI). El objetivo de este ejercicio es el manejo y manipulación de una gran cantidad de datos. Finalmente, el tercer ejercicio presenta la programación de un algoritmo de búsqueda que se utiliza en la discriminación de modelos en diseños de experimentos. En este ejercicio, los cálculos son más intensivos por lo que busca medir la capacidad y velocidad de cómputo de los lenguajes. 

A continuación, se comienza este trabajo con la presentación de \textsf{Julia}. 



