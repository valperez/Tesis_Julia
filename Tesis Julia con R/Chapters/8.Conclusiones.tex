\chapter{Conclusiones}

El objetivo de este trabajo fue hacer una comparación entre \textsf{Julia}, \textsf{R} y \textsf{Python}. En primer lugar y dado que \textsf{Julia} es un lenguaje todavía muy desconocido se dió una breve introducción a los básicos de \textsf{Julia} y lo indispensable para entender esta tesis. Se asumió que el lector ya conoce \textsf{R} y \textsf{Python} por lo que se expusieron de manera más superficial. 

La segunda parte de la tesis fueron los ejercicios que, se piensa, aportan conocimiento sobre el mejor uso de los tres lenguajes. En el primer ejercicio se tomaron datos de NIST para hacer el ajuste de un polinomio. Los datos son sensibles y el problema presentó un reto numéricamente. Se realizaron cuatro intentos de resolución en \textsf{Julia} de los cuales solo uno proporcionó el resultado correcto. La solución en \textsf{R} y, sobretodo, en \textsf{Python} fue más rápida y sencilla. 

\valinline{Falta decir un poco más, siento}
El segundo ejercicio fue ajustar una regresión lineal a datos obtenidos con el CENSO 2020. Si bien \textsf{Julia} pudo hacer los filtros de manera adecuada la sintaxis de los comandos es un poco extraña y no da pie al fácil entendimiento de la instrucción. El objetivo fue comparar y evaluar el manejo de datos en los tres lenguajes. Los tres lenguajes mostraron hacer un análisis rápido y certero para todas las pruebas que se hicieron.

El tercer ejercicio fue el más retador teóricamente y en práctica. Un diseño de experimentos se hace para obtener información sobre los factores que afectan o no un resultado. El algoritmo que se utilizó fue el propuesto por Meyer para elegir los ensayos extras necesarios en caso de que los primeros no dieran suficiente información sobre el modelo que describe el experimento. La fórmula para el valor MD es compleja y se tiene que hacer hasta que el algoritmo termine (alrededor de 70 veces). Por lo tanto, este ejercicio fue una prueba de rapidez de cálculos intensivos y es el que más resalta las habilidades y funciones de \textsf{Julia}. El mismo algoritmo fue hecho en los tres lenguajes, pero \textsf{Julia} fue el más rápido de todos. 

En mi opinión, \textsf{Julia} no busca ser la copia de \textsf{R} o \textsf{Python} en análisis y manejo de datos (el dichoso data science). \textsf{Julia} está hecho para ser un lenguaje que realiza simulaciones y cálculos complejos con un alto rendimiento. \textsf{Julia} tiene mucho que ofrecer, pero un análisis de datos fácil y intuitivo no es una de esas cosas. Su uso va mucho más allá y se enfoca en ser el mejor lenguaje para hacer estadística bayesiana, análisis epistemológicos y análisis de sistemas dinámicos. 

Habiendo dicho lo anterior, este trabajo se puede extender de muchas maneras. La primera que se viene a la mente es enfocarse en otras áreas de las matemáticas y seguir con la comparación. 
\valinline{Se me fue la otra que queria decirrrrrrrrrrrrr}



