\chapter{R} \label{cap_R}

\say{Ross Ihaka y Robert Gentleman, del departamento de Estadística de Auckland University en Nueva Zelanda estaban interesados en el cómputo estadístico y reconocieron la necesidad de un mejor ambiente de cálculo del que tenían. Ninguno de los productos comerciales les convencía, por lo que decidieron desarrollar uno propio}, \cite{laberintos_historiaR}. 

\textsf{R} nació de la necesidad de tener una transición de usuario a desarrollador. \cite{peng_programming} explica que los creadores buscaron desarrollar un lenguaje que podría usarse para hacer análisis de datos de manera interactiva y, que además, fuera capaz de escribir programas más largos. Una de las principales cualidades de \textsf{R} es la facilidad para crear gráficos bien diseñados y con calidad de publicación que pueden incluir símbolos matemáticos y fórmulas en caso de ser necesarios \citep{pagina_r}. 

La experiencia de la autora es que \textsf{R} es uno de los lenguajes más sencillos para aprender a trabajar métodos estadísticos computacionales. La sencillez de su sintaxis permite que incluso un usuario nuevo navegue de manera fluida por el código. Su popularidad en la comunidad científica ha impulsado el desarrollo del lenguaje y la creación de todo tipo de materiales de apoyo. Las razones anteriores son el motivo por el cual se decidió incluir el lenguaje en esta tesis. 

Debido a la popularidad ya mencionada, en este trabajo se parte de la primicia de que el lector ya cuenta con los conocimientos básicos para entender el código que se presenta. El avanzado desarrollo del lenguaje permitió que los ejercicios de este trabajo se hicieran con pocas funciones. En el último proyecto se utiliza un paquete ya programado, mientras que los primeros dos se enfocan en la función \texttt{lm} explicada a continuación. 

\section{Función lm} \label{explicacion_lm}

La función \texttt{lm} es usada para analizar modelos lineales. La manera de llamarla es con el comando \texttt{lm(formula, data, subset, weights, na.action, method = 'qr', model = TRUE, x = FALSE, y = FALSE, qr = TRUE, singular.ok = TRUE, contrasts = NULL, offset, ...)}. Con esto se puede observar que la función cuenta con muchos argumentos que la hacen muy versátil. De hecho, en el capítulo \ref{cap_polinomios} es necesario modificar el argumento \texttt{tol} para poder resolver el problema de manera exitosa. 

El modelo de regresión lineal multivariada debe tener como el argumento de \texttt{formula} una ecuación de la forma

\begin{lstlisting}[language=R]
	y ~ x_1 + x_2 + ... + x_k
\end{lstlisting}

Es decir, se indica la variable de respuesta $y$ seguido de una virgulilla y después los $n$ predictores que se estén utilizando. Si se tratara de escribir la \texttt{formula} como ecuación, la virgulilla juega el papel de un signo de igualdad.

La función \texttt{lm} también permite analizar modelos con variables cuyo grado es mayor a uno. En este caso, se debe utilizar la función llamada interpretación inhibida, mejor conocida como \texttt{I(...)}. El argumento \texttt{formula} comienza como en el caso anterior, con la variable $y$ y la virgulilla. Después, se eleva el regresor $x$ a la potencia $k$ dentro de la función \texttt{I(...)}. Por ejemplo, si se quisiera ajustar el modelo  $y = \beta_0 + \beta_1 x + \beta_2 x^{2}$ el comando es

\begin{lstlisting}[language=R]
	y ~ x + I(x^2)
\end{lstlisting}

La función \texttt{I(...)} se utiliza para cambiar la clase de un objeto para indicar que el objeto debería ser tratado de la forma 'como si fuera'. Esta instrucción se usa especialmente para los operadores especiales de fórmula como es el caso de \texttt{$\wedge$}. En el ejemplo anterior, la función de interpretación inhibida da la instrucción de tomar $x^{2}$ como una variable y no como la interacción de segundo orden de $x$. 

Por otro lado, uno de los argumentos que tiene la función \texttt{lm} es \texttt{tol}, la tolerancia del ajuste. En este caso, la tolerancia determina si las columnas de una matriz son linealmente independientes o no. Cuando se tiene una matriz con valores muy pequeños (como es el caso del ejercicio presentado en el capítulo \ref{cap_polinomios}) se necesita la tolerancia para determinar cuando un valor se considera como cero. En dicho ejercicio el argumento \texttt{tol} tuvo que ser modificado para lograr el resultado correcto. Usualmente este parámetro no necesita ser cambiado, pero es útil tomarlo en cuenta para ocasiones donde los datos son extremadamente sensibles y se busca un ajuste preciso. 

La segunda parte de la tesis comienza en el siguiente capítulo. En esta parte se exponen tres ejercicios diferentes en los tres lenguajes descritos en la primera parte (\textsf{R, Julia} y \textsf{Python}). El primer ejercicio es el ajuste de un modelo lineal de grado diez con datos sensibles. En este ejercicio se mide la precisión numérica de los cálculos de los tres lenguajes. El segundo ejercicio es el ajuste de modelos lineales de distintos órdenes usando grandes cantidades de datos. El objetivo fue ilustrar y comparar el manejo y análisis de datos. Finalmente, el tercer ejercicio es sobre la discriminación de modelos en diseños de experimentos. El punto de comparación fue la rapidez en la que los lenguajes hacen una gran cantidad de cálculos intensivos. 


