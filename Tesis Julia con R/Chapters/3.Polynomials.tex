\chapter{\textit{Polynomials}}

Polynomials es un paquete de Julia que proporciona arimética básica, integración, diferenciación, evaluación, raíces para polinomios univariados \cite{poly_manual}. Para instalar el paquete hay que seguir los comandos \ref{instalacion_paquete}. Esta sección va a explicar como ajustar un polinomio a un conjunto de datos. 

\section{El problema}
Supongamos que tenemos un conjunto de datos con solamente dos variables $x$ y $y$. Buscamos ajustarlos a un polinomio de grado $k$. Es decir, buscamos ajustar los datos al modelo

\begin{equation} \label{eq_polinomio}
    \begin{aligned}
    y = \sum_{j = 0}^{k} \beta_j x^j + \epsilon
    \end{aligned}
\end{equation}

El problema consiste en encontrar los mejores coeficientes que cumplan la ecuación anterior. Para hacerlo, el paquete Polynomials utiliza una implementación del método de Gauss-Newton para resolver un problema de mínimos cuadrados no lineal. \val{Ya sé que dice que es para problemas no lineales y el mío es un problema lineal pero es el único que funciona}


\section{Método Gauss-Newton}
\valinline{Supongo que lo tengo que explicar???}



\section{Ejemplo}
Para probar la función \texttt{fit} del paquete Polynomials elegí utilizar el conjunto de datos llamado \textit{Filip} del Laboratorio de Información Tecnológica perteneciente al Instituto Nacional de Estándares y Tecnología (NIST por sus siglas en inglés). Los datos y el resultado de la estimación de coeficientes están en \url{https://www.itl.nist.gov/div898/strd/lls/data/LINKS/DATA/Filip.dat}. 
\\
El código en Julia para obtener el resultado es el siguiente: 

\begin{minted}{julia}
    using Polynomials
    filip = CSV.read("filip_data.csv", DataFrame)
    coef_filip = CSV.read("coeficientes_filip.csv", DataFrame)
    x = filip.x
    y = filip.y
    as = coef_filip.coeficiente
    p = Polynomials.fit(x,y, 10)
\end{minted}

Los datos con las 82 observaciones en forma de pares ordenados es \texttt{filip\_data.csv}, mientras que los datos con los resultados de los coeficientes dados por el NIST es \texttt{coeficientes\_filip.csv}. Dichos documentos fueron obtenidos de la página mencionada anteriormente. 
\\
El resultado obtenido en Julia es \val{Lo voy a dejar asi aunque se ve pésimo, solo es para que se vean todos los coeficientes}

\begin{tcolorbox}
    \begin{verbatim}
    -1467.4896770751313 - 2772.179712060834x
    - 2316.3711834983233x^2 - 1127.973991485157x^3 -
    354.478249903365x^4 - 75.12420525407894x^5 -
    10.875318557918678x^6 - 1.0622150384213702x^7 -
    0.06701911888098762x^8 - 0.0024678109131484544x^9 -
    4.0296254716574806e-5x^10
    \end{verbatim}
\end{tcolorbox}

\valinline{OJO, no me da el error de la estimación, ni valor t ni nada}

Para comparar los resultados con las estimaciones reales de los coeficientes, uso el siguiente código de Julia:

\begin{minted}{julia}
    as = coef_filip.coeficiente
    q = Polynomial(as)
    maximum(abs(p(x) - q(x)) for x in range(0,1, length=1000))
\end{minted}

Es decir, creo un polinomio con las estimaciones de los coeficientes de NIST y después hago la diferencia en valor absoluto con la estimación que yo obtuve. El resultado es

\begin{tcolorbox}
    \begin{verbatim}
        0.0003723332774825394
    \end{verbatim}
\end{tcolorbox}

Por lo tanto, en la peor estimación de $\beta_i$, Julia acierta hasta 3 dígitos. 

