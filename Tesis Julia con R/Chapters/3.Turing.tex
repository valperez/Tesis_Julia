\chapter{Turing}
Explicar porque usar el paquete llamado Turing. Según sus creadores, es a general-purpose probabilistic programming language for robust, efficient Bayesian inference and decision making. Current features include:

General-purpose probabilistic programming with an intuitive modelling interface;
Robust, efficient Hamiltonian Monte Carlo (HMC) sampling for differentiable posterior distributions;
Particle MCMC sampling for complex posterior distributions involving discrete variables and stochastic control flow; and
Compositional inference via Gibbs sampling that combines particle MCMC, HMC and random-walk MH (RWMH).

\section{Modelos}
Introducción a hacer modelos en Julia con las distribuciones que ya vimos

\subsection{Eficiencia del modelo}
Poner que si le dices al modelo que esperas que te salga, le toma menos tiempo

\subsubsection{Ejemplo}
Ejemplo donde muestre que efectivamente toma menos tiempo

\section{Función sample}
Explicar como usar sample para hacer que tu modelo corra muchas veces

\subsection{HMC}
Explicar para que sirve HMC

\subsection{NUTS}
Explicar para que sirve NUTS

\subsection{Iteraciones y Burnin}

\section{Pruebas de convergencia}
¿Cómo sé que lo que salió, es correcto?
Buen link: https://juliahub.com/ui/Packages/MCMCChain/93CWR/0.2.3

\subsection{Paquete MCMCChains}
Como se usa para visualizar si las simulaciones están bien

\subsection{Gelman}
Que es y como se usa en Julia

\subsection{Heidel}
Que es y como se usa en Julia

\subsection{Effective Sample Size (ESS)}
Que es y como se usa en Julia

\section{Ejemplo}