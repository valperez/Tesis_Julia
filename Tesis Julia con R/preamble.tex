\usepackage[paperwidth=17cm, paperheight=22.5cm, bottom=2.5cm, right=2.5cm]{geometry}

% El borde inferior puede parecerles muy amplio a la vista. Les recomiendo hacer una prueba de impresión antes para ajustarlo

\usepackage{amssymb,amsmath,amsthm} % Símbolos matemáticos
\usepackage[spanish,mexico,es-tabla]{babel}
\usepackage[utf8]{inputenc} % Acentos y otros símbolos 
\usepackage{enumerate}
\usepackage{optidef}
\usepackage{hyperref} % Hipervínculos en el índice
\usepackage[spanish]{cleveref}
\usepackage{graphicx}
\usepackage[usenames,dvipsnames]{xcolor} % Color
%\usepackage{subfig} % Subfiguras
\usepackage{listings}%Para los códigos de MALTAB, ver documentación de matlab-prettifier
\usepackage[framed]{matlab-prettifier}
\usepackage[linesnumbered,lined,boxruled,spanish,onelanguage]{algorithm2e}
\usepackage{titling}
\usepackage{amsfonts}
\usepackage[thinc]{esdiff}
\spanishdecimal{.}
\definecolor{itamgreen}{RGB}{0,104,83}
\hypersetup{
    colorlinks=true,
    linkcolor=itamgreen,
    filecolor=itamgreen,      
    urlcolor=itamgreen,
    citecolor=itamgreen
}
\urlstyle{same}


\usepackage{translations}
\graphicspath{{Imagenes/}} % En qué carpeta están las imágenes

\DeclareTranslationFallback{rank}{rank}
\DeclareTranslation{english}{rank}{rank}
\DeclareTranslation{spanish}{rank}{rango}

\DeclareMathOperator{\rank}{\GetTranslation{rank}}


% Para eliminar guiones y justificar texto
\tolerance=1
\emergencystretch=\maxdimen
\hyphenpenalty=10000
\hbadness=10000

\linespread{1.25} % Asemeja el interlineado 1.5 de Word

\let\oldfootnote\footnote % Deja espacio entre el número del pie de página y el inicio del texto
\renewcommand\footnote[1]{%
\oldfootnote{\hspace{0.05mm}#1}}

\newcommand{\bigO}[1]{\ensuremath{\mathop{}\mathopen{}\mathcal{O}\mathopen{}\left(#1\right)}}

\newcommand\smallO[1]{
    \mathchoice
    {% mode \displaystyle
      \scriptstyle\mathcal{O}\left(#1\right)
    }
    {% mode \textstyle
      \scriptstyle\mathcal{O}\left(#1\right)
    }
    {% mode \scriptstyle
      \scriptscriptstyle\mathcal{O}\left(#1\right)
    }
    {% mode \scriptscriptstyle
      \scalebox{0.7}{$\scriptscriptstyle\mathcal{O}$}\left(#1\right)
    }
}
  

\renewcommand{\thefootnote} {\textcolor{Black}{\arabic{footnote}}} % Súperindice a color negro

\setlength{\footnotesep}{0.75\baselineskip} % Espaciado entre notas al pie

\usepackage{fnpos} % Footnotes al final de pág.

\usepackage[justification=centering, font=bf, labelsep=period, skip=5pt]{caption} % Centrar captions de tablas y ponerlas en negritas

\newcommand{\imagesource}[1]{{\footnotesize Fuente: #1}}

\usepackage{tabularx} % Big tables
\usepackage{adjustbox}
\usepackage{longtable}


\usepackage{float} % Float tables

\usepackage{pgfplots} % Gráficas
\pgfplotsset{compat=newest}
\pgfplotsset{width=7.5cm}
\pgfkeys{/pgf/number format/1000 sep={}}

\newtheorem{theorem}{Teorema}[chapter]
\newtheorem{corollary}{Corolario}[theorem]
\newtheorem{lemma}[theorem]{Lema}
\newtheorem{definition}{Definición}
\newtheorem{plain}{Proposición}
\newtheorem{remark}{Corolario}[plain]

\newcommand{\Tr}[1]{\operatorname{Tr}\left(#1\right)}
\newcommand{\diag}[1]{\operatorname{diag}\left(#1\right)}
\newcommand{\frNm}[1]{\left\|#1\right\|_{\mathrm{F}}}

\SetKwBlock{Loop}{Repetir}{fin}

\usepackage{afterpage}
\newcommand\myemptypage{
    \null
    \thispagestyle{empty}
    \newpage
    }
    
    
\usepackage{todonotes}
%HAY Q BORRAR ESTOS COMMANDS EVENTUALMENTE
\newcommand{\val}[1]{\todo[color=blue!20!white]{\textbf{Vale:} #1}}
\newcommand{\valinline}[1]{\todo[inline,color=blue!20!white]{\textbf{Vale:} #1}}

\usepackage{natbib}

\usepackage{tcolorbox}

\usepackage{dirtytalk}

\newcommand{\dquote}[1]{``#1''}

\usepackage{minted}